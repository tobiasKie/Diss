\chapter{Konzeption}
\label{chap:char_konzept}
Im Folgenden Kapitel, wird die dieser Arbeit zugrunde liegende Aufgabe in kleinere Einzelne Arbeitspakete unterteilt. Bei herkömmlichen magnetischen Bewegungstracking Systemen wird zunächst jedes Element einzeln lokalisiert und im Anschluss zu einer kinematischen Kette zusammengefügt. Dies führt zu einer zweistufigen Struktur, einer vorigen Lokalisierung und einem anschließenden Fitting. 

Der Aufbau des Systems soll aus einem Handschuh bestehen, bei dem jedes kinematische Element mit einem Sensor ausgestattet wird. Auf dem Handgelenk wird eine 3D-Spule angebracht, die ein künstliches magnetisches Feld erzeugen kann. Im gesamten Raum werden ortsfeste Sensoren angebracht.

\section{Einordnung und Unterteilung in Aufgabenpakete}
\label{sec:Konzept_einteilung}
Die Lokalisierungsaufgabe soll für eine bessere Effizienz in zwei kleinere Aufgaben aufgeteilt werden. Zum einen eine relative Lokalisierung, die die Handhaltung oder Haltung der kinematischen Kette bestimmt. Zum anderen wird eine absolute Lokalisierung benötigt, welche die Position und Orientierung der 3D-Spule innerhalb des Messvolumens bestimmt.

\subsection{Relative Lokalisierung}
\label{sec:RelLoc}

\subsection{Absolute Lokalisierung}
\label{sec:AbsLoc}

		    
 
