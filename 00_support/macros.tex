%========================================================================================
% Titel, Name
%========================================================================================
\newcommand{\tit}{{Signalverarbeitung f\"ur das Bewegungstracking von kinematischen Ketten}}
\newcommand{\subtit}{}
\newcommand{\ath}{Vorname Nachname}
\newcommand{\locdate}{Kiel, 2024}

%========================================================================================
% Mathe Formatierungen
%========================================================================================
\newcommand{\transp}{^\textrm{\footnotesize{T}}}                              % Transpose
\newcommand{\herm}{^\textrm{\footnotesize{H}}}                                % Hermitrian
\newcommand{\expect}[1]{\textrm{E}\left\{\,\displaystyle{{#1}}\,\right\}}     % Erwartungswert
\newcommand{\mse}[1]{\textrm{E}\left\{\left|{#1}(n)\right|^{2}\right\}}       % Mittlerer Qudratische Erwartungswert
\newcommand{\E}{\operatorname{E}}

\newcommand{\tindex}[1]{_{\textrm{\footnotesize{#1}}}}                        % Text tiefgestellt, nicht kursiv
\newcommand{\exponent}[1]{^{\textrm{\footnotesize{#1}}}}                      % Text tiefgestellt, nicht kursiv
\newcommand{\tindexTwo}[2]{_{\footnotesize{#1}\textrm{\footnotesize{#2}}}}    % Text tiefgestelt, erst kursiv, dann nicht kursiv
\newcommand{\subit}[1]{_{\mathit{#1}}}                                        % Text tiefgestellt, kursiv

\newcommand{\Exp}[1]{\mathrm{E}\!\left\{{#1}\right\}}                         % Ewartungswert
\newcommand{\ExpLeft}{\mathrm{E}\!\left\{\right\}}                            % Ewartungswert Klammer links
\newcommand{\ExpRight}{\mathrm{E}\!\right\}}                                  % Ewartungswert Klammer rechts
\newcommand{\ExpSmall}[1]{\mathrm{E}$\,$\!\{{#1}\}}                           % Ewartungswert, klein

\newcommand{\abs}[1]{\lvert {#1} \rvert}                                      % Betrag
\newcommand{\absSq}[1]{\abs{#1}\exponent{2}}                                  % Betragsquadrat
\newcommand{\norm}[1]{\lVert {#1} \rVert}                                     % Norm
\newcommand{\normSq}[1]{\norm{#1}\exponent{2}}                                % Quadratnorm
\newcommand{\Smo}[1]{\overline{#1}}                                           % Geglättete Variable
\newcommand{\SmoSq}[1]{\Smo{#1\exponent{2}}}                                  % Geglättetes Betragsquadrat
\newcommand{\unl}[1]{\underline{#1}}                                          % Variable unterstrichen
\newcommand{\round}[1]{\ensuremath\left\lfloor#1\right\rceil}                 % Runden einer Variable
\newcommand{\eExpOmega}{\big( e^{\footnotesize{\, j \Omega}} \big) }          % e hoch Omega
\newcommand{\freqMu}{\big( \mu \big) }

\newcommand{\vectorize}[1]{\textrm{VEC}\left\{\displaystyle{{#1}}\right\}}    % Vektorisieren
\newcommand{\diagonalize}[1]{\textrm{DIAG}\left\{\displaystyle{{#1}}\right\}} % Diagonaliseren

%========================================================================================
% Matrizen und Vektoren
%========================================================================================
\newcommand{\bvec}[1]{\mbox{\boldmath ${#1}$}}                 % Vektoren fett
\newcommand{\bmat}[1]{\mbox{\boldmath \underline{${#1}$}}}     % Matrizen fett, roman

%========================================================================================
% Misc commands
%========================================================================================
\newcommand{\red}[1]{\textcolor{red}{[#1]}}        % Notes and ToDos
\newcommand{\qm}[1]{``#1''}                        % Anführungszeichen
\newcommand{\engl}{engl.}                          % Englische abgekürzt
\newcommand{\OverSqrtHz}{/$\sqrt{\text{Hz}}$}      % Wurzel-Hertz

%========================================================================================
% General defines
%========================================================================================
\newcommand{\ffsAxes}{0.7}
\newcommand{\ffsLegend}{0.7}
\newcommand{\ffsNumbers}{0.7}
\newcommand{\ffsFormula}{0.7}
\newcommand{\ffsFormulaSFrac}{0.8}
\newcommand{\ffsExp}{0.5}	

%========================================================================================
% Legacy defines 
%========================================================================================
\newcommand{\tsup}[1]{^{\text{#1}}}
\newcommand{\tidx}[1]{_{\text{{#1}}}}

\newcommand{\argx}[1]{\ensuremath{\text{arg}\left\{#1\right\}}}
\newcommand{\absx}[1]{\ensuremath{\left|#1\right|}}
\newcommand{\absxnormal}[1]{\ensuremath{|#1|}}
\newcommand{\lgx}[1]{\ensuremath{\text{lg} \left( #1\right) }}
\newcommand{\maxx}[1]{\ensuremath{\text{max}\left\{#1\right\}}}
\newcommand{\minx}[1]{\ensuremath{\text{min}\left\{#1\right\}}}
\newcommand{\normx}[1]{\ensuremath{\left\|#1\right\|}}

%========================================================================================
% Abkürzungen
%========================================================================================
\newcommand{\AD}{AD}
\newcommand{\DA}{DA}

\newcommand{\EBASNR}{EBASNR}
\newcommand{\SNR}{$\text{SNR}$}
\newcommand{\DNR}{$\text{DNR}$}
\newcommand{\LOD}{$\text{LOD}$}

\newcommand{\MEMS}{MEMS}

\newcommand{\MVDR}{MVDR}

\newcommand{\NLMS}{NLMS}
\newcommand{\LMS}{LMS}
\newcommand{\AP}{AP}
\newcommand{\RLS}{RLS}

%========================================================================================
% Allgemeine Größen
%========================================================================================

%% Allgemein
\newcommand{\fs}{\ensuremath{f\tidx{s}}}

%% Indizes
\newcommand{\td}{\ensuremath{n}}	   % Digital time
\newcommand{\ta}{\ensuremath{t}}    % Analog time
\newcommand{\fb}{\ensuremath{\mu}}  % Frequency bin
\newcommand{\fri}{\ensuremath{k}}   % Frame index
