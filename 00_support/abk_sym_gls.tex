%========================================================================================
% Abkürzungen
%========================================================================================
\newglossaryentry{wE}{type=abk,sort={wE},name={w.E.},description={willkürliche Einheit}}
\newglossaryentry{Engl}{type=abk,sort={Engl},name={engl.},description={englisch}}

\newglossaryentry{AD}{type=abk,sort={AD},name={\AD},description={Analog-Digital}}
\newglossaryentry{DA}{type=abk,sort={DA},name={\DA},description={Digital-Analog}}

\newglossaryentry{EBASNR}{type=abk,sort={EBASNR},name={\EBASNR},description={geschätztes, biomagnetisches und gemitteltes \SNR{}  (\engl{}: \textit{Estimated Biomagnetic Averaged Signal-to-Noise Ratio})}}
\newglossaryentry{SNR}{type=abk,sort={SNR},name={\SNR},description={Signal-zu-Rausch-Verhältnis (\engl{}: \textit{Signal-to-Noise Ratio})}}
\newglossaryentry{DNR}{type=abk,sort={DNR},name={\DNR},description={Störungs-zu-Rausch-Verhältnis (\engl{}: \textit{Distortion-to-Noise Ratio})}}
\newglossaryentry{LOD}{type=abk,sort={LOD},name={\LOD},description={Detektionslimit (\engl{}: \textit{Limit of Detection})}}

\newglossaryentry{MEMS}{type=abk,sort={MEMS},name={\MEMS},description={mikro-elektromechanische Systeme (\engl{}: \textit{Microelectromechanical Systems})}}

\newglossaryentry{MVDR}{type=abk,sort={MVDR},name={\MVDR},description={Minimum Variance Distortionless Beamformer (\engl{}: \textit{Minimum Variance Distortionless Beamformer })}}

\newglossaryentry{NLMS}{type=abk,sort={NLMS},name={\NLMS{}-Algorithmus},description={normalisierter stochastischer Gradientenalgorithmus (\engl{}: \textit{Normalized Least Mean Square Algorithm})}}
\newglossaryentry{LMS}{type=abk,sort={LMS},name={\LMS{}-Algorithmus},description={stochastischer Gradientenalgorithmus (\engl{}: \textit{Least Mean Square Algorithm})}}
\newglossaryentry{AP}{type=abk,sort={AP},name={\AP},description={Affine Projektion (\engl{}: \textit{Affine Projection})}}
\newglossaryentry{RLS}{type=abk,sort={RLS},name={\RLS{}-Algorithmus},description={rekursives stochastisches Gradientenverfahren (\engl{}: \textit{Recursive Least Squares Algorithm})}}

\newglossaryentry{AlN}{type=abk,sort={AlN},name={AlN},description={Aliminiumnitrid}}
\newglossaryentry{FeCoSiB}{type=abk,sort={FeCoSiB},name={FeCoSiB},description={metallisches Glas (Eisen-Cobalt-Silizium-Bor)}}
\newglossaryentry{PZT}{type=abk,sort={PZT},name={PZT},description={Blei-Zirkonat-Titanat}}
\newglossaryentry{Au}{type=abk,sort={Au},name={Au},description={Gold}}
\newglossaryentry{Cr}{type=abk,sort={Cr},name={Cr},description={Chrom}}
\newglossaryentry{Cu}{type=abk,sort={Cu},name={Cu},description={Kupfer}}
\newglossaryentry{Ta}{type=abk,sort={Ta},name={Ta},description={Tantal}}
\newglossaryentry{MnIr}{type=abk,sort={MnIr},name={MnIr},description={Mangan-Iridium}}

%========================================================================================
% Notation
%========================================================================================
\newglossaryentry{nE}{type=notation,sort={g},name={$\E{x(n)}$},description={Erwartungswert von $x(n)$} }

\newglossaryentry{nvec}{type=notation,sort={a},name={\ensuremath{\bvec{x}}},description={Vektor (fettgedruckt, klein)} }
\newglossaryentry{nmat}{type=notation,sort={a},name={$\bvec{X}$},description={Matrix (fettgedruckt, groß)} }
\newglossaryentry{nskalar}{type=notation,sort={a},name={$x$},description={Skalar, zumeist zeitabhängig (nicht fettgedruckt, klein)} }
\newglossaryentry{nSkalar}{type=notation,sort={a},name={$X$},description={Skalar, zumeist frequenzabhängig (nicht fettgedruckt, groß)} }

\newglossaryentry{nj}{type=notation,sort={b},name={$j$},description={imaginäre Einheit, $j^2=-1$} }
\newglossaryentry{njRe}{type=notation,sort={b},name={$\Re{\left\{ z \right\} }$},description={Realteil der komplexen Zahl $z$} }
\newglossaryentry{njIm}{type=notation,sort={b},name={$\Im{\left\{ z \right\} }$},description={Imaginärteil der komplexen Zahl $z$} }
\newglossaryentry{njAbs}{type=notation,sort={b},name={$\absx{z}$},description={Betrag der komplexen Zahl $z$} }
\newglossaryentry{njArg}{type=notation,sort={b},name={$\argx{z}$},description={Argument der komplexen Zahl $z$} }
\newglossaryentry{njKonj}{type=notation,sort={b},name={$z^{*}$},description={konjugierte komplexe Zahl zur komplexen Zahl $z$} }

\newglossaryentry{nbvecTransp}{type=notation,sort={c},name={$\bvec{x}\transp{}, \bvec{X}\transp{}$},description={transponierter Vektor/Matrix} }
\newglossaryentry{nbvecHerm}{type=notation,sort={c},name={$\bvec{x}\herm{}, \bvec{X}\herm{}$},description={hermitischer Vektor/Matrix, $ \bvec{X}\herm{} = \left(\bvec{X}^{*} \right)\transp{}$} }

\newglossaryentry{nlog10}{type=notation,sort={d},name={$\lgx{z}$},description={Logarithmus der reellen Zahl $z$ zur Basis 10} }

\newglossaryentry{nhadamard}{type=notation,sort={e},name={$\bvec{x} \circ \bvec{y}$},description={Hadamard-Produkt, elementweise Multiplikation} }

\newglossaryentry{nnorm}{type=notation,sort={e},name={$\normx{\bvec{x}}$},description={Norm des Vektors} }

%\newglossaryentry{nDFT}{type=notation,sort={f},name={$\DFTmath{}\left\{ \cdot{} \right\}$},description={\DFT{} der Ordnung \Ndft{}} }
%\newglossaryentry{nDSTFT}{type=notation,sort={f},name={$\DSTFTmath{}\left\{ \cdot{} \right\}$},description={\DSTFT{} der Ordnung \Ndft{}} }


\newglossaryentry{nmax}{type=notation,sort={h},name={$\maxx{x(n)}$},description={Maximum von $x(n)$} }
\newglossaryentry{nmin}{type=notation,sort={h},name={$\minx{x(n)}$},description={Minimum von $x(n)$} }

%========================================================================================
% Symbols
%========================================================================================
%% Indizes
\newglossaryentry{td}{type=lat,sort={na},name={\td},description={Zeitindex}  }
\newglossaryentry{ta}{type=lat,sort={ta},name={\ta},description={Zeit}  }
\newglossaryentry{fb}{type=gre,sort={mau},name={\fb},description={Frequenzstützstelle}  }
\newglossaryentry{fri}{type=lat,sort={ka},name={\fri},description={Blockindex}  }

\newglossaryentry{fs}{type=lat,sort={fas},name={\fs},description={Abtastrate}  }



%========================================================================================
% Ende
%========================================================================================
\glsaddall